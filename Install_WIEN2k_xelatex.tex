\documentclass[10pt,a4paper]{article}

%%%%%%%%%%%%%%%%% CJK 中文版面控制  %%%%%%%%%%%%%%%%%%%%%%%%%%%%%%
%\usepackage{CJK} % CTEX-CJK 中文支持                            %
\usepackage{xeCJK} % seperate the english and chinese		 %
%\usepackage{CJKutf8} % Texlive 中文支持                          %
\usepackage{CJKnumb} %中文序号                                   %
\usepackage{indentfirst} % 中文段落首行缩进                      %
%\setlength\parindent{22pt}       % 段落起始缩进量               %
\renewcommand{\baselinestretch}{1.2} % 中文行间距调整            %
\setlength{\textwidth}{16cm}                                     %
\setlength{\textheight}{24cm}                                    %
\setlength{\topmargin}{-1cm}                                     %
\setlength{\oddsidemargin}{0.1cm}                                %
\setlength{\evensidemargin}{\oddsidemargin}                      %
%%%%%%%%%%%%%%%%%%%%%%%%%%%%%%%%%%%%%%%%%%%%%%%%%%%%%%%%%%%%%%%%%%

\usepackage{amsmath,amsthm,amsfonts,amssymb,bm}          %数学公式
\usepackage{mathrsfs}                                    %英文花体

\usepackage{xcolor}                                        %使用默认允许使用颜色

\usepackage{fontspec} % use to set font
\setCJKmainfont{SimSun}
\XeTeXlinebreaklocale "zh"  % Auto linebreak for chinese
\XeTeXlinebreakskip = 0pt plus 1pt % Auto linebreak for chinese

\usepackage{longtable}                                   %使用长表格

%%%%%%%%%%%%%%%%%%%%%%%%%  参考文献引用 %%%%%%%%%%%%%%%%%%%%%%%%%%%
%%尽量使用 BibTeX(含有超链接,数据库的条目URL即可)                %
%%%%%%%%%%%%%%%%%%%%%%%%%%%%%%%%%%%%%%%%%%%%%%%%%%%%%%%%%%%%%%%%%%%

\usepackage[numbers,sort&compress]{natbib} %紧密排列             %
\usepackage[sectionbib]{chapterbib}        %每章节单独参考文献   %
\usepackage{hypernat}                                                                         %
\usepackage[bookmarksopen=true,pdfstartview=FitH,CJKbookmarks]{hyperref}              %
\hypersetup{bookmarksnumbered,colorlinks,linkcolor=green,citecolor=blue,urlcolor=red}         %
%参考文献含有超链接引用时需要下列宏包,注意与natbib有冲突        %
%\usepackage[dvipdfm]{hyperref}                                  %
%\usepackage{hypernat}                                           %
\newcommand{\upcite}[1]{\hspace{0ex}\textsuperscript{\cite{#1}}} %

%%%%%%%%%%%%%%%%%%%%%%%%%%%%%%%%%%%%%%%%%%%%%%%%%%%%%%%%%%%%%%%%%%%%%%%%%%%%%%%%%%%%%%%%%%%%%%%
%\AtBeginDvi{\special{pdf:tounicode GBK-EUC-UCS2}} %CTEX用dvipdfmx的话,用该命令可以解决      %
%						   %pdf书签的中文乱码问题		      %
%%%%%%%%%%%%%%%%%%%%%%%%%%%%%%%%%%%%%%%%%%%%%%%%%%%%%%%%%%%%%%%%%%%%%%%%%%%%%%%%%%%%%%%%%%%%%%%

%%%%%%%%%%%%%%%%%%%%%  % EPS 图片支持  %%%%%%%%%%%%%%%%%%%%%%%%%%%
\usepackage{graphicx}                                            %
\usepackage{xcolor}                                        %使用默认允许使用颜色
%%%%%%%%%%%%%%%%%%%%%%%%%%%%%%%%%%%%%%%%%%%%%%%%%%%%%%%%%%%%%%%%%%


\begin{document}
%\CJKindent     %在CJK环境中,中文段落起始缩进2个中文字符
\graphicspath{{figure/}}
%
\renewcommand{\abstractname}{\small{\CJKfamily{hei} 摘\quad 要}} %\CJKfamily{hei} 设置中文字体,字号用\big \small来设
\renewcommand{\refname}{\centering\CJKfamily{hei} 参考文献}
%\renewcommand{\figurename}{\CJKfamily{hei} 图.}
\renewcommand{\figurename}{{\bf Fig}.}
%\renewcommand{\tablename}{\CJKfamily{hei} 表.}
\renewcommand{\tablename}{{\bf Tab}.}

%将图表的Caption写成 图(表) Num. 格式
\makeatletter
\long\def\@makecaption#1#2{%
  \vskip\abovecaptionskip
  \sbox\@tempboxa{#1. #2}%
  \ifdim \wd\@tempboxa >\hsize
    #1. #2\par
  \else
    \global \@minipagefalse
    \hb@xt@\hsize{\hfil\box\@tempboxa\hfil}%
  \fi
  \vskip\belowcaptionskip}
\makeatother

\newcommand{\keywords}[1]{{\hspace{0\ccwd}\small{\CJKfamily{hei} 关键词:}{\hspace{2ex}{#1}}\bigskip}}

%%%%%%%%%%%%%%%%%%中文字体设置%%%%%%%%%%%%%%%%%%%%%%%%%%%
%默认字体 defalut fonts \TeX 是一种排版工具 \\		%
%{\bfseries 粗体 bold \TeX 是一种排版工具} \\		%
%{\CJKfamily{song}宋体 songti \TeX 是一种排版工具} \\	%
%{\CJKfamily{hei} 黑体 heiti \TeX 是一种排版工具} \\	%
%{\CJKfamily{kai} 楷书 kaishu \TeX 是一种排版工具} \\	%
%{\CJKfamily{fs} 仿宋 fangsong \TeX 是一种排版工具} \\	%
%%%%%%%%%%%%%%%%%%%%%%%%%%%%%%%%%%%%%%%%%%%%%%%%%%%%%%%%%

%\addcontentsline{toc}{section}{Bibliography}

%-------------------------------The Title of The Paper-----------------------------------------%
\title{WIEN2k软件的安装}
%----------------------------------------------------------------------------------------------%

%----------------------The Authors and the address of The Paper--------------------------------%
\author{
\small
%Author1, Author2, Author3\footnote{Communication author's E-mail} \\    %Authors' Names	       %
\small
%(The Address,City Post code)						%Address	       %
}
\date{}					%if necessary					       %
%----------------------------------------------------------------------------------------------%
\maketitle

%-------------------------------------------------------------------------------The Abstract and the keywords of The Paper----------------------------------------------------------------------------%
%\begin{abstract}
%The content of the abstract
%\end{abstract}

%\keywords {Keyword1; Keyword2; Keyword3}
%-----------------------------------------------------------------------------------------------------------------------------------------------------------------------------------------------------%

%----------------------------------------------------------------------------------------The Body Of The Paper----------------------------------------------------------------------------------------%
%Introduction
\textrm{WIEN2k}软件是在\textrm{Linux~}系统下安装、运行的第一原理计算软件,其编译、安装主要参考\textrm{WIEN2k\_Users-Guide}\upcite{WIEN2K-UG_2001}的第\textrm{11}章\textrm{Installation and Dimensioning}。

除了\textrm{Linux}操作系统,编译、安装环境建议如下,(以下括号中的版本号都是我当前使用的版本):\\
\begin{itemize}
	\item csh或tcsh
	\item intel编译器 (ifort / icc) (版本 13.0.0) + MKL库函数
	\item 2017版需要\textrm{intel\_\textcolor{red}{2015}}及更高版本的编译器和库函数
	\item $\ast$\textrm{VASP}-\textcolor{red}{5.4.4}需要\textrm{intel\_\textcolor{red}{composer\_xe\_2013\_sp1.3.174}}及更高版本的编译器和库函数\\
		否则编译文件\underline{\textcolor{blue}{~wave.f90~}}时会提示:~“\textcolor{red}{指针数组连续未声明}”:\\
		\textcolor{brown}{wave.F(669): error \#8378: Pointer array is not contiguous unless it is declared CONTIGUOUS.$\quad[\mathrm{CPTWFP}]$}\\
		\textcolor{brown}{WUP\%CPTWFP$=>$ W\%CPTWFP(:,:,:,1)}\\
		\textcolor{brown}{-{}-{}-{}-{}-{}-{}-{}-{}-{}-{}-{}-{}-{}-{}-{}-{}-{}-{}-{}-{}-{}-{}\^{}}
\end{itemize}

为了支持mpi版本的WIEN2k,还需要如下支持:
\begin{itemize}
	\item mpi编译器 openmpi (版本 openmpi-1.8.4) 或 mpich (版本 mpich-3.1.4) 
	\item fftw库 (版本 fftw-3.3.4)
\end{itemize}
\textcolor{red}{注意:}以下安装假定系统和环境设置的默认编译器和编译环境为 \textrm{intel} 编译器环境和 \textrm{mkl} 库(即 \~\//.bashrc 中的环境设置为):

\textcolor{blue}{\#\#\#\#\#\#\#\#\#\#\#\#\#\#\#\# INTEL\_compiler \#\#\#\#\#\#\#\#\#\#\#\#\#\#\#\#\#\#\#}

\textcolor{blue}{. \underline{/home/soft/intel/Compiler/11.0/083}/bin/intel64/ifortvars\_intel64.sh}

\textcolor{blue}{. \underline{/home/soft/intel/Compiler/11.0/083}/bin/intel64/iccvars\_intel64.sh}

\textcolor{blue}{\#\#\#\#\#\#\#\#\#\#\#\#\#\#\# INTEL\_mkl \#\#\#\#\#\#\#\#\#\#\#\#\#\#\#\#\#\#\#}

\textcolor{blue}{export LD\_LIBRARY\_PATH=\underline{/home/soft/intel/mkl/10.1.2.024}/lib/em64t/:\$LD\_LIBRARY\_PATH}

或者如果直接安装了INTEL编译器完整版

\textcolor{blue}{\#\#\#\#\#\#\#\#\#\#\#\#\#\#\#\# INTEL\_compiler\_mkl \#\#\#\#\#\#\#\#\#\#\#\#\#\#\#\#\#\#\#}

\textcolor{blue}{source \underline{你的INTEL安装目录}/bin/compilervars.sh intel64}

\section{openmpi的安装:}
tar -xvzf openmpi-1.8.4.tar.gz

cd openmpi-1.8.4

./configure -\/-prefix=\underline{你的MPI安装目录} CC=icc CXX=icpc F77=ifort FC=ifort -\/--enable-static

make \&\& make install

安装完毕后,新增 \~\//.bashrc 中的环境变量设置:

\textcolor{blue}{\#\#\#\#\#\#\#\#\#\#\#\#\#\# OPENMP \#\#\#\#\#\#\#\#\#\#\#}

\textcolor{blue}{export PATH=\underline{你的MPI安装目录}/bin:\$PATH}


\textcolor{red}{注意:}如果安装mpich,操作完全类似

\textcolor{red}{注意:}查看有关mpi本身指向的编译器等信息的命令 \underline{\textcolor{blue}{mpiexec -info}}\\
\underline{\textcolor{blue}{mpif90 --version}}

\section{SCAlapack安装}
唯一需要注意的是\textrm{SCAlapack}安装2.0以上版本(自带\textrm{BLACS}),需要\textrm{LAPACK},\textrm{BLAS}库支持

\section{fftw的安装:}
\textcolor{blue}{一般地,\textrm{Intel~}的\textrm{MKL~}自带了\textrm{FFTW~}的库函数,其中
\begin{itemize}
	\item 头文件位于\;\$MKLROOT/include/fftw
	\item 静态\textrm{fortran~}库函数位于\;\$MKLROOT/interfaces/fftw3xf/libfftw3xf\_intel.a
	\item 静态\textrm{c~}库函数位于\;\$MKLROOT/interfaces/fftw3xc/libfftw3xc\_intel.a
\end{itemize}}

\textcolor{purple}{注意:~}\$MKLROOT/include/fftw的文件包括\\
\textcolor{green}{fftw3.f\;\;  fftw3\_mkl.f\;\;   fftw3\_mkl.h\;\;  fftw3-mpi\_mkl.h\;\;  fftw.h \;\;    fftw\_threads.h\;\;  rfftw\_mpi.h \\
	fftw3.h\;\;  fftw3\_mkl\_f77.h\;\;  fftw3-mpi.h\;\;  fftw\_f77.i \;\;      fftw\_mpi.h \;\; rfftw.h  \;\;       rfftw\_threads.h
}\\
因此如果源代码包括 \underline{\textcolor{green}{fftw3.f03\;\; fftw3.h\;\; fftw3-mpi.f03\;\; fftw3-mpi.h}}等这样的文件

\textbf{\textcolor{red}{则必须安装\textrm{fftw~}库函数}}


tar -xvzf fftw-3.3.4.tar.gz

cd fftw-3.3.4

./configure -\/-prefix=\underline{你的FFTW安装目录} CC=icc MPICC=mpicc F77=ifort FC=ifort MPILIBS=-I\underline{你的MPI安装目录}/include -\/-enable-mpi \textcolor{brown}{-\/-enable-float -\/-enable-threads -\/-enable-sse -\/-enable-sse2}

make \&\& make install

如果要生成\textrm{fftw/fftw\_mpi}动态库,用如下的设置

./configure -\/-prefix=\underline{你的FFTW安装目录} \textcolor{red}{CC=icc CXX=icpc CFLAGS=-fPIC} F77=ifort FC=ifort \textcolor{red}{-\/-enable-shared}\;\;\;\textcolor{violet}{仅出现动态库}

./configure -\/-prefix=\underline{你的FFTW安装目录} CC=icc MPICC=mpicc F77=ifort FC=ifort MPILIBS=-I\underline{你的MPI安装目录}/include -\/-enable-mpi \textcolor{red}{-\/-enable-shared}

如果要生成包含\textrm{\textcolor{red}{fftw3f/fftw3f\_mpi/\_thread}}等动态库,用如下的设置

./configure -\/-prefix=\underline{你的FFTW安装目录} CC=icc MPICC=mpicc F77=ifort FC=mpif90 MPILIBS=-I\underline{你的MPI安装目录}/include -\/-enable-mpi \textcolor{red}{-\/-enable-shared }\textcolor{brown}{-\/-enable-float -\/-enable-threads -\/-enable-sse -\/-enable-sse2}

如果用\textcolor{red}{\textrm{gcc/gfortran}}编译器编译

./configure -\/-prefix=\underline{你的FFTW安装目录} \textcolor{red}{CC=gcc MPICC=mpicc F77=gfortran FC=mpif90} MPILIBS=-I\underline{你的MPI安装目录}/include -\/-enable-mpi \textcolor{red}{-\/-enable-shared} \textcolor{brown}{-\/-enable-float -\/-enable-threads -\/-enable-sse -\/-enable-sse2}

\section{库函数检查的有关命令}
\textbf{nm}

\textcolor{red}{-D或--dynamic}\;\;显示动态符号。该任选项仅对于动态目标(例如特定类型的共享库)有意义

\textcolor{red}{-g或--extern-only}\;\;仅显示外部符号

\textcolor{red}{-u或--undefined-only}\;\;仅显示没有定义的符号(那些外部符号)

\textbf{ldd}

\textbf{ldd}命令用于判断某个可执行的 binary 文件含有什么\textcolor{blue}{动态函式库}

\textcolor{red}{-d --data-relocs}\;\; 执行符号重部署,并报告缺少的目标对象(只对ELF格式适用)

\textcolor{red}{-r --function-relocs}\;\; 对目标对象和函数执行重新部署,并报告缺少的目标对象和函数(只对ELF格式适用)

\textbf{readelf}

\textbf{readelf}命令查看共享库的依赖库(\textrm{NEED})和搜索名(\textrm{SONAME})

\textcolor{red}{-d}\;\;查看动态库的真实名字

\section{WIEN2k的安装}
选定安装目录(没有则新建一个目录),将\textrm{WIN2k}安装文件包安放在该目录下。根据\textrm{WIEN2k\_Users-Guide}\upcite{WIEN2K-UG_2001}的第\textrm{11}章\textrm{Installation and Dimensioning}说明,\textrm{WIEN2k}的安装如下:

tar -xvf WIEN2k-version.tar

gunzip *.gz

./expand\_lapw

\textcolor{red}{建议:}新增 \~\//.bashrc 的环境变量设置:

\textcolor{blue}{MKLPATH = \underline{你的\textrm{INTEL\_MKL}库目录}}

接下来的编译安装主要通过脚本 \textrm{siteconfig\_lapw}完成,其中涉及的最重要的编译参数设置如下:

FOPT = \textcolor{red}{-FR -mp1 -w -prec\_div -pc80 -pad -ip -DINTEL\_VML -traceback -assume buffered\_io -DFFTW3 -I\underline{你的FFTW安装目录}/include}

LDFLAGS = \textcolor{red}{\$(FOPT) -L\${MKLPATH} -pthread}

R\_LIBS = \textcolor{red}{-lfftw3\; -lmkl\_lapack95\_lp64\; -lmkl\_intel\_lp64\; -lmkl\_intel\_thread\; -lmkl\_core\; -openmp\; -L\underline{你的FFTW安装目录}/lib\; -lpthread}

RP\_LIBS = \textcolor{red}{-lfftw3\_mpi\; \${MKLPATH}/libmkl\_scalapack\_lp64.a\; -Wl,--start-group\; \${MKLPATH}/libmkl\_sequential.a\; \${MKLPATH}/libmkl\_blacs\_openmpi\_lp64.a\; \${MKLPATH}/libmkl\_core.a\; -Wl,--end-group\/ \$(R\_LIBS)}

\textcolor{red}{特别地},$\star$对\textrm{LAPW0}的\textrm{mpi}并行:\\
 \textcolor{blue}{RP\_LIBS =-lfftw3\_mpi\; \${MKLPATH}/libmkl\_scalapack\_lp64.a\; -Wl,--start-group\; \${MKLPATH}/libmkl\_sequential.a\; \${MKLPATH}/libmkl\_core.a\; \${MKLPATH}/libmkl\_blacs\_openmpi\_lp64.a\; -Wl,--end-group\/ \$(R\_LIBS)}

\textcolor{red}{注意:}如果是用mpich,将 \textrm{ RP\_LIBS } 中的选项\\
\${MKLPATH}/libmkl\_blacs\_openmpi\_lp64.a 替换为\\
\${MKLPATH}/libmkl\_blacs\_intelmpi\_lp64.a \\
\textcolor{red}{或者},将 \textrm{ RP\_LIBS } 中的\textrm{RP\_LIBS}选项设置为\\
RP\_LIBS = \textcolor{red}{-lfftw3\_mpi\; -lmkl\_scalapack\_lp64\; -lmkl\_sequential\; -lmkl\_blacs\_intelmpi\_lp64\; \${MKLPATH}/libmkl\_blas95\_lp64.a\; -L\underline{你的MPICH安装目录}/lib\; -lmpich\; -lmpichf90\/ \$(R\_LIBS)}

\section{安装后环境设置}
安装后的环境配置主要通过脚本 \textrm{userconfig\_lapw}完成,我自己选择的文本编辑器是 \textrm{vim}(默认的是 \textrm{emacs} ),其余的都选用默认值。

完成环境配置后,如有必要,还可以通过编辑 \~\//.bashrc 修改有关参数:

\textcolor{red}{注意:}如选用\textrm{mpi}并行版本,建议对 \~\//.bashrc 作下列修改
\begin{itemize}
	\item 注释以下这行:\\
\textcolor{blue}{\#ulimit -s unlimited}
	\item	修改parallel\_options\\
\textcolor{blue}{setenv WIEN\_MPIRUN ``mpirun -machinefile \_HOSTS\_ -np \_NP\_ \_EXEC\_''}
\end{itemize}

\section{配置web界面}
用\textrm{root}用户打开\textrm{apache}服务

service apache2 start

在普通用户下执行

\textrm{w2web}

将打开\textrm{WIEN2k}默认的\textrm{7890}端口作为\textrm{WIEN2k}的\textrm{web}界面

\section{.machines文件的编写}
\textrm{.machines}文件指定了并行计算所使用的计算资源,因此需要平衡计算资源的负载平衡。\textrm{.machines}文件中每一项详细说明参见\textrm{WIEN2k\_Users-Guide}\upcite{WIEN2K-UG_2001}的第\textrm{5}章之\textrm{5.5 Running programs in parallel mode}。
根据\textrm{.machines}文件不同决定进行\textrm{k-point}或\textrm{mpi}并行计算:\\\\
\textrm{k-point}并行的\textrm{.machines}:\\
granularity:1 \\
1:node31:1 \#格式:1:指定节点名:1 (\textrm{k-point}并行方式)\\
1:node31:1 \\
1:node32:1 \\
1:node32:1 \\
lapw0:node31:2 node32:2 \# 指定lapw0并行方式:lapw0:指定节点名:核数 n \\
extrafine:1 \\\\
\textrm{mpi}并行的\textrm{.machines}:\\
granularity:1 \\
1:node31:2 \# 格式:1:指定节点名:核数 n (\textrm{mpi}并行)\\
1:node32:2 \\
lapw0:node31:2 node32:2 \# 指定lapw0并行方式:lapw0:指定节点名:核数 n \\
extrafine:1


\section{采用作业调度提交作业}
在pbs作业管理系统中,手动编辑\textrm{.machines}实现WIEN2k的并行费时费力,这里提供一个脚本(\textrm{wien2k.pbs})可以根据分配的计算资源自动生成\textrm{.machines}。

\textcolor{red}{注意:该脚本使用时需要根据计算环境修改计算参数}\\
cat wien2k.pbs \\
\#\#\#\#\#\#\#\#\#\#\#\#\#\#\#\#\#\#\#\#\#\#\#\#\#\#\#\#\#\#\#\#\#\#\#\#\#\#\#\#\#\#\#\#\#\#\#\#\#\#\#\#\#\#\#\\%\#\#\#\#\#\#\#\#\#\#\#\#\#\#\#\#\#\#\#\#\\
\# \#\\
\# Script for submitting parallel wien2k jobs to Dawning cluster. \#\\
\# \#\\
\#\#\#\#\#\#\#\#\#\#\#\#\#\#\#\#\#\#\#\#\#\#\#\#\#\#\#\#\#\#\#\#\#\#\#\#\#\#\#\#\#\#\#\#\#\#\#\#\#\#\#\#\#\#\#\\%\#\#\#\#\#\#\#\#\#\#\#\#\#\#\#\#\#\#\#\#\\
\#\#\#\#\#\#\#\#\#\#\#\#\#\#\#\#\#\#\#\#\#\#\#\#\#\#\#\#\#\#\#\#\#\#\#\#\#\#\#\#\#\#\#\#\#\#\#\#\#\#\#\#\#\#\#\\%\#\#\#\#\#\#\#\#\#\#\#\#\#\#\#\#\#\#\#\#\\
\# Lines that begin with \# PBS are PBS directives (not comments).\\
\# True comments begin with "\# " (i.e., \# followed by a space).\\
\#\#\#\#\#\#\#\#\#\#\#\#\#\#\#\#\#\#\#\#\#\#\#\#\#\#\#\#\#\#\#\#\#\#\#\#\#\#\#\#\#\#\#\#\#\#\#\#\#\#\#\#\#\#\#\\%\#\#\#\#\#\#\#\#\#\#\#\#\#\#\#\#\#\#\#\#\\
\#PBS -S /bin/bash\\
\#PBS -N TiO2\\
\#PBS -j oe\\
\#PBS -l nodes=1:ppn=8\\
\#PBS -V\\
\#\#\#\#\#\#\#\#\#\#\#\#\#\#\#\#\#\#\#\#\#\#\#\#\#\#\#\#\#\#\#\#\#\#\#\#\#\#\#\#\#\#\#\#\#\#\#\#\#\#\#\#\#\#\#\\%\#\#\#\#\#\#\#\#\#\#\#\#\#\#\#\#\#\#\#\#\\
\#\#\#\#\#\#\#\#\#\#\#\#\#\#\#\#\#\#\#\#\#\#\#\#\#\#\#\#\#\#\#\#\#\#\#\#\#\#\#\#\#\#\#\#\#\#\#\#\#\#\#\#\#\#\#\\%\#\#\#\#\#\#\#\#\#\#\#\#\#\#\#\#\#\#\#\\
\# -S: shell the job will run under\\
\# -o: name of the queue error filename\\
\# -j: merges stdout and stderr to the same file\\
\# -l: resources required by the job: number of nodes and processors per node\\
\# -l: resources required by the job: maximun job time length\\
\#\#\#\#\#\#\#\#\#\#\#\#\#\#\#\#\#\#\#\#\#\#\#\#\#\#\#\#\#\#\#\#\#\#\#\#\#\#\#\#\#\#\#\#\#\#\#\#\#\#\#\#\#\#\#\\%\#\#\#\#\#\#\#\#\#\#\#\#\#\#\#\#\#\#\#\#\\
\#\#\#\#\#\#\#\#\# parallel mode is mpi/kpoint \#\#\#\#\#\#\#\#\#\#\#\#\\
PARALLEL=mpi // 表示采用\textrm{mpi}并行或\textrm{k}点并行\\
echo \$PARALLEL\\
\#\#\#\#\#\#\#\#\#\#\#\#\#\#\#\#\#\#\#\#\#\#\#\#\#\#\#\#\#\#\#\#\#\#\#\#\#\#\#\#\#\#\#\#\#\#\#\#\#\#\#\#\#\#\#\#\\%\#\#\#\#\#\#\#\#\#\#\#\#\#\#\#\#\#\#\#\\
NP=`cat \${PBS\_NODEFILE} | wc -l` \\
NODE\_NUM=`cat \$PBS\_NODEFILE|uniq |wc -l` \\
NP\_PER\_NODE=`expr \$NP / \$NODE\_NUM` \\
username=`whoami` \\
export WIENROOT=\underline{你的WIEN2k安装目录} \\
export PATH=\$PATH:\$WIENROOT:. \\
WIEN2K\_RUNDIR=/scratch/\${username}.\${PBS\_JOBID} \\
export SCRATCH=\${WIEN2K\_RUNDIR} \\
\# creat scratch dir \\
if [ ! -a \$WIEN2K\_RUNDIR ]; then \\
echo "Scratch directory \$WIEN2K\_RUNDIR created." \\
mkdir -p \$WIEN2K\_RUNDIR \\
fi \\
cd \$PBS\_O\_WORKDIR \\
\#\#\#\#\#\#\#\#\# Creating .machines \#\#\#\#\#\#\#\#\#\#\#\# \\
case \$PARALLEL in \\
mpi) \\
echo "granularity:1" $>$ .machines \\
for i in `cat \$PBS\_NODEFILE | uniq ` \\
do \\
echo "1:"\$i":"\$NP\_PER\_NODE $>>$ .machines \\
done \\
printf "lapw0:" $>>$ .machines \\
\#\#\#\#\#\#\#\#\# \textrm{lapw0} 用 \textrm{mpi} 并行 \#\#\#\#\#\#\#\#\#\#\#\# \\
for i in `cat \${PBS\_NODEFILE} | uniq` \\
do \\
printf \$i:\$NP\_PER\_NODE" " $>>$ .machines \\
done \\
\#\#\#\#\#\#\#\#\#\#\#\#\#\#\#\#\#\#\#\#\#\#\#\#\#\#\#\#\#\#\#\#\# \\
\#\#\#\#\#\#\#\#\# \textrm{lapw0} 用 \textrm{mpi} 并行报错的算例用以下 mpi\_error\_lapw0 \#\#\#\#\#\#\#\#\#\#\#\# \\
\# printf `cat \${PBS\_NODEFILE}| uniq | head -1`:1 $>>$ .machines \\
\#\#\#\#\#\#\#\#\#\#\#\#\# end \#\#\#\#\#\#\#\#\#\#\#\#\#\#\#\#\# \\
printf "\\n" $>>$ .machines \\ 
echo "extrafine:1" $>>$ .machines \\
;; \\
kpoint) \\
echo "granularity:1" $>$ .machines \\
for i in `cat \$PBS\_NODEFILE` \\
do \\
echo "1:"\$i":" 1 $>>$ .machines \\
done \\
printf "lapw0:" $>>$ .machines \\
\#\#\#\#\# \textrm{lapw0} 用 \textrm{mpi} 并行 \#\#\#\#\#\#\#\#\#\#\#\#\# \\
for i in `cat \${PBS\_NODEFILE} | uniq` \\
do \\
printf $i:$NP\_PER\_NODE" " $>>$ .machines \\
done \\
\#\#\#\#\#\#\#\#\#\#\#\#\#\#\#\#\#\#\#\#\#\#\#\#\#\#\#\#\#\#\#\#\# \\
\#\#\#\# \textrm{lapw0} 用 \textrm{mpi} 并行报错的算例用以下 \textrm{mpi\_error\_lapw0} \#\#\#\#\#\#\#\# \\
\# printf `cat \${PBS\_NODEFILE} | uniq | head -1`:1 $>>$ .machines \\
\#\#\#\#\#\#\#\#\#\#\#\#\# end \#\#\#\#\#\#\#\#\#\#\#\#\#\#\#\#\# \\
printf "\\n" $>>$ .machines \\
echo "extrafine:1" $>>$ .machines \\
;; \\
esac \\
\#\#\#\#\#\#\#\#\#\#\#\#\#\#\#\#\# end creating \#\#\#\#\#\#\#\#\#\#\#\#\#\#\#\#\#\#\#\# \\
\#\#\#\#\#\#\# Run the parallel executable "WIEN2K" \#\#\#\#\#\#\#\#\# \\
instgen\_lapw \\
init\_lapw -b \\
clean -s \\
echo "\#\#\#\#\#\#\#\#\#\#\#\#\#\#\#\#\#\# start time is `date` \#\#\#\#\#\#\#\#\#\#\#\#\#\#\#\#\#\#\#\#\#\#\#\#" \\
run\_lapw -p \\
echo "\#\#\#\#\#\#\#\#\#\#\#\#\#\#\#\#\#\# end time is `date` \#\#\#\#\#\#\#\#\#\#\#\#\#\#\#\#\#\#\#\#\#\#\#\#" \\
rm -rf \$WIEN2K\_RUNDIR \\
\#\#\#\#\#\#\#\#\#\#\#\#\#\#\#\#\#\#\#\#\#\#\#\# END \#\#\#\#\#\#\#\#\#\#\#\#\#\#\#\#\#\#\#\#\#\#\#\#

%一般需要修改的地方已用红字标出
该脚本可以实现算例的初始化,必须在存在*.struct的前提下进行。

%-------------------The Figure Of The Paper------------------
%\begin{figure}[h!]
%\centering
%\includegraphics[height=3.35in,width=2.85in,viewport=0 0 400 475,clip]{PbTe_Band_SO.eps}
%\hspace{0.5in}
%\includegraphics[height=3.35in,width=2.85in,viewport=0 0 400 475,clip]{EuTe_Band_SO.eps}
%\caption{\small Band Structure of PbTe (a) and EuTe (b).}%(与文献\cite{EPJB33-47_2003}图1对比)
%\label{Pb:EuTe-Band_struct}
%\end{figure}

%-------------------The Equation Of The Paper-----------------
%\begin{equation}
%\varepsilon_1(\omega)=1+\frac2{\pi}\mathscr P\int_0^{+\infty}\frac{\omega'\varepsilon_2(\omega')}{\omega'^2-\omega^2}d\omega'
%\label{eq:magno-1}
%\end{equation}

%\begin{equation} 
%\begin{split}
%\varepsilon_2(\omega)&=\frac{e^2}{2\pi m^2\omega^2}\sum_{c,v}\int_{BZ}d{\vec k}\left|\vec e\cdot\vec M_{cv}(\vec k)\right|^2\delta [E_{cv}(\vec k)-\hbar\omega] \\
% &= \frac{e^2}{2\pi m^2\omega^2}\sum_{c,v}\int_{E_{cv}(\vec k=\hbar\omega)}\left|\vec e\cdot\vec M_{cv}(\vec k)\right|^2\dfrac{dS}{\nabla_{\vec k}E_{cv}(\vec k)}
% \end{split}
%\label{eq:magno-2}
%\end{equation}

%-------------------The Table Of The Paper----------------------
%\begin{table}[!h]
%\tabcolsep 0pt \vspace*{-12pt}
%\caption{The representative $\vec k$ points contributing to $\sigma_2^{xy}$ of interband transition in EuTe around 2.5 eV.}
%\label{Table-EuTe_Sigma}
%\begin{minipage}{\textwidth}
%%\begin{center}
%\centering
%\def\temptablewidth{1.01\textwidth}
%\rule{\temptablewidth}{1pt}
%\begin{tabular*} {\temptablewidth}{@{\extracolsep{\fill}}cccccc}

%-------------------------------------------------------------------------------------------------------------------------
%&Peak (eV)  & {$\vec k$}-point            &Band{$_v$} to Band{$_c$}  &Transition Orbital
%Components\footnote{波函数主要成分后的括号中,$5s$、$5p$和$5p$、$4f$、$5d$分别指碲和铕的原子轨道。} &Gap (eV)   \\ \hline
%-------------------------------------------------------------------------------------------------------------------------
%&2.35       &(0,0,0)         &33$\rightarrow$34    &$4f$(31.58)$5p$(38.69)$\rightarrow$$5p$      &2.142   \\% \cline{3-7}
%&       &(0,0,0)         &33$\rightarrow$34    &$4f$(31.58)$5p$(38.69)$\rightarrow$$5p$      &2.142   \\% \cline{3-7}
%-------------------------------------------------------------------------------------------------------------------------

%\end{tabular*}
%\rule{\temptablewidth}{1pt}\\
%%\end{center}
%\end{minipage}
%\end{table}

%-------------------The Long Table Of The Paper--------------------
%\begin{small}
%%\begin{minipage}{\textwidth}
%%\begin{longtable}[l]{|c|c|cc|c|c|} %[c]指定长表格对齐方式
%\begin{longtable}[c]{|c|c|p{1.9cm}p{4.6cm}|c|c|}
%\caption{Assignment for the peaks of EuB$_6$}
%\label{tab:EuB6-1}\\ %\\长表格的caption中换行不可少
%\hline
%%
%--------------------------------------------------------------------------------------------------------------------------------
%\multicolumn{2}{|c|}{\bfseries$\sigma_1(\omega)$谱峰}&\multicolumn{4}{c|}{\bfseries部分重要能带间电子跃迁\footnotemark}\\ \hline
%\endfirsthead
%--------------------------------------------------------------------------------------------------------------------------------
%%
%\multicolumn{6}{r}{\it 续表}\\
%\hline
%--------------------------------------------------------------------------------------------------------------------------------
%标记 &峰位(eV) &\multicolumn{2}{c|}{有关电子跃迁} &gap(eV)  &\multicolumn{1}{c|}{经验指认} \\ \hline
%\endhead
%--------------------------------------------------------------------------------------------------------------------------------
%%
%\multicolumn{6}{r}{\it 续下页}\\
%\endfoot
%\hline
%--------------------------------------------------------------------------------------------------------------------------------
%%
%%\hlinewd{0.5$p$t}
%\endlastfoot
%--------------------------------------------------------------------------------------------------------------------------------
%%
%% Stuff from here to \endlastfoot goes at bottom of last page.
%%
%--------------------------------------------------------------------------------------------------------------------------------
%标记 &峰位(eV)\footnotetext{见正文说明。} &\multicolumn{2}{c|}{有关电子跃迁\footnotemark} &gap(eV) &\multicolumn{1}{c|}{经验指认\upcite{PRB46-12196_1992}}\\ \hline
%--------------------------------------------------------------------------------------------------------------------------------
%
%     &0.07 &\multicolumn{2}{c|}{电子群体激发$\uparrow$} &--- &电子群\\ \cline{2-5}
%\raisebox{2.3ex}[0pt]{$\omega_f$} &0.1 &\multicolumn{2}{c|}{电子群体激发$\downarrow$} &--- &体激发\\ \hline
%--------------------------------------------------------------------------------------------------------------------------------
%
%     &1.50 &\raisebox{-2ex}[0pt][0pt]{20-22(0,1,4)} &2$p$(10.4)4$f$(74.9)$\rightarrow$ &\raisebox{-2ex}[0pt][0pt]{1.47} &\\%\cline{3-5}
%     &1.50$^\ast$ & &2$p$(17.5)5$d_{\mathrm E}$(14.0)$\uparrow$ & &4$f$$\rightarrow$5$d_{\mathrm E}$\\ \cline{3-5}
%     \raisebox{2.3ex}[0pt][0pt]{$a$} &(1.0$^\dagger$) &\raisebox{-2ex}[0pt][0pt]{20-22(1,2,6)} &\raisebox{-2ex}[0pt][0pt]{4$f$(89.9)$\rightarrow$2$p$(18.7)5$d_{\mathrm E}$(13.9)$\uparrow$}\footnotetext{波函数主要成分后的括号中,2$s$、2$p$和5$p$、4$f$、5$d$、6$s$分别指硼和铕的原子轨道;5$d_{\mathrm E}$、5$d_{\mathrm T}$分别指铕的(5$d_{z^2}$,5$d_{x^2-y^2}$和5$d_{xy}$,5$d_{xz}$,5$d_{yz}$)轨道,5$d_{\mathrm{ET}}$(或5$d_{\mathrm{TE}}$)则指5个5$d$轨道成分都有,成分大的用脚标的第一个字母标示;2$ps$(或2$sp$)表示同时含有硼2$s$、2$p$轨道成分,成分大的用第一个字母标示。$\uparrow$和$\downarrow$分别标示$\alpha$和$\beta$自旋电子跃迁。} &\raisebox{-2ex}[0pt][0pt]{1.56} &激子跃迁。 \\%\cline{3-5}
%     &(1.3$^\dagger$) & & & &\\ \hline
%--------------------------------------------------------------------------------------------------------------------------------

%     & &\raisebox{-2ex}[0pt][0pt]{19-22(0,0,1)} &2$p$(37.6)5$d_{\mathrm T}$(4.5)4$f$(6.7)$\rightarrow$ & & \\\nopagebreak %\cline{3-5}
%     & & &2$p$(24.2)5$d_{\mathrm E}$(10.8)4$f$(5.1)$\uparrow$ &\raisebox{2ex}[0pt][0pt]{2.78} &a、b、c峰可能 \\ \cline{3-5}
%     & &\raisebox{-2ex}[0pt][0pt]{20-29(0,1,1)} &2$p$(35.7)5$d_{\mathrm T}$(4.8)4$f$(10.0)$\rightarrow$ & &包含有复杂的\\ \nopagebreak%\cline{3-5}
%     &2.90 & &2$p$(23.2)5$d_{\mathrm E}$(13.2)4$f$(3.8)$\uparrow$ &\raisebox{2ex}[0pt][0pt]{2.92} &强激子峰。$^{\ast\ast}$\\ \cline{3-5}
%$b$  &2.90$^\ast$ &\raisebox{-2ex}[0pt][0pt]{19-22(0,1,1)} &2$p$(33.9)4$f$(15.5)$\rightarrow$ & &B2$s$-2$p$的价带 \\ \nopagebreak%\cline{3-5}
%     &3.0 & &2$p$(23.2)5$d_{\mathrm E}$(13.2)4$f$(4.8)$\uparrow$ &\raisebox{2ex}[0pt][0pt]{2.94} &顶$\rightarrow$B2$s$-2$p$导\\ \cline{3-5}
%     & &12-15(0,1,2) &2$p$(39.3)$\rightarrow$2$p$(25.2)5$d_{\mathrm E}$(8.6)$\downarrow$ &2.83 &带底跃迁。\\ \cline{3-5}
%     & &14-15(1,1,1) &2$p$(42.5)$\rightarrow$2$p$(29.1)5$d_{\mathrm E}$(7.0)$\downarrow$ &2.96 & \\\cline{3-5}
%     & &13-15(0,1,1) &2$p$(40.4)$\rightarrow$2$p$(28.9)5$d_{\mathrm E}$(6.6)$\downarrow$ &2.98 & \\ \hline
%--------------------------------------------------------------------------------------------------------------------------------
%%\hline
%%\hlinewd{0.5$p$t}
%\end{longtable}
%%\end{minipage}{\textwidth}
%%\setlength{\unitlength}{1cm}
%%\begin{picture}(0.5,2.0)
%%  \put(-0.02,1.93){$^{1)}$}
%%  \put(-0.02,1.43){$^{2)}$}
%%\put(0.25,1.0){\parbox[h]{14.2cm}{\small{\\}}
%%\put(-0.25,2.3){\line(1,0){15}}
%%\end{picture}
%\end{small}

%-----------------------------------------------------------------------------------------------------------------------------------------------------------------------------------------------------%


%--------------------------------------------------------------------------The Biblography of The Paper-----------------------------------------------------------------%
%\newpage																				%
%-----------------------------------------------------------------------------------------------------------------------------------------------------------------------%
%\begin{thebibliography}{99}																		%
%%\bibitem{PRL58-65_1987}H.Feil, C. Haas, {\it Phys. Rev. Lett.} {\bf 58}, 65 (1987).											%
%\end{thebibliography}																			%
%-----------------------------------------------------------------------------------------------------------------------------------------------------------------------%
%																					%
\phantomsection\addcontentsline{toc}{section}{Bibliography}	 %直接调用\addcontentsline命令可能导致超链指向不准确,一般需要在之前调用一次\phantomsection命令加以修正	%
\bibliography{ref/Myref}																			%
\bibliographystyle{ref/mybib}																		%
%  \nocite{*}																				%
%-----------------------------------------------------------------------------------------------------------------------------------------------------------------------%

\clearpage     %\end{CJK} 前加上\clearpage是CJK的要求
\end{document}
